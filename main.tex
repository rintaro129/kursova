\documentclass[a4paper, 14pt]{article}
\usepackage{fontspec, extsizes, geometry, setspace, titlesec, fancyhdr, graphicx, float, setspace, caption, array, tabularx, ulem, indentfirst, ragged2e, caption, appendix}
\usepackage[main=ukrainian, english]{babel} % підтримка мов
\usepackage[dotinlabels]{titletoc} 
\setmainfont[Ligatures=TeX]{Times New Roman}
\geometry{a4paper,left=3cm,top=2cm,bottom=2cm,right=1.5cm} % cтавить береги та формат а4
\def\changemargin#1#2{\list{}{\rightmargin#2\leftmargin#1}\item[]}\let\endchangemargin=\endlist % зручна команда для виставлення відступів
\makeatletter\newcommand\Dotfill{\leavevmode\leaders\hb@xt@0.5em{\hss.\hss}\hfill}\makeatother % команда для ставлення точок
\let\stdsection\section\renewcommand\section{\newpage\stdsection} % Новая секція -> нова сторінка
\addto\captionsukrainian{\renewcommand{\figurename}{Рисунок}} % підпис картинок не Рис. а Рисунок
\titleformat{\section}{\filcenter}{\bfseries{\thesection} }{0pt}{\bfseries\MakeUppercase} % Форматування заголовків всіх розділів
% \renewcommand{\thesubsection}{\arabic{section}.\arabic{subsection}} % Чиним номер подраздела
\titleformat{\subsection}{}{{\thesubsection} }{0pt}{} % нумерація підрозділів
\titlespacing*{\subsection}{\parindent}{\baselineskip}{\baselineskip} % рядок перед та пілся підрозділу та підрозділ з абзацу
\titleformat{\subsubsection}{}{{\thesubsubsection} }{0pt}{} % нумерація підпідрозділів
\titlespacing*{\subsubsection}{\parindent}{\baselineskip}{\baselineskip} % рядок перед та пілся підпідрозділу та підпідрозділ з абзацу
\titlespacing*{\section}{0pt}{\baselineskip}{\baselineskip} % рядок перед та пілся розділу
\captionsetup{labelsep=endash} % "Рис. 1 - замість Рис. 1:"
\fancyhf{}\renewcommand{\headrulewidth}{0pt}\newcommand{\changefont}{\fontsize{14}{14}\selectfont}\fancyhead[R]{\changefont \thepage}\fancypagestyle{plain}{\fancyhf{}\fancyhead[R]{\changefont \thepage}\renewcommand{\headrulewidth}{0pt}\renewcommand{\footrulewidth}{0pt}}\pagestyle{fancy} %номер страницы справа сверху на всех страничках [это ужас]
\linespread{1.43} % Інтервал абзацу полуторний
% \renewcommand{\contentsname}{ЗМІСТ} %изменяем название странички с содержанием
\def\numberline#1{#1. } % Фикс чтобы названия не налезали друг на друга в содержании
\titlecontents{section}[0pt]{\normalfont}{{\thecontentslabel} }{}{\Dotfill \contentspage} % оформление разделов, точек в содержании
\titlecontents{subsection}[15pt]{\normalfont}{{\thecontentslabel} }{}{\Dotfill \contentspage} % оформление подразделов, точек в содержании
\titlecontents{subsubsection}[30pt]{\normalfont}{{\thecontentslabel} }{}{\Dotfill \contentspage} % оформление подподразделов, точек в содержании
\counterwithin{figure}{section} % нумерация картинок с номером раздела
\counterwithin{table}{section} % нумерация таблиц с номером раздела
\usepackage[none]{hyphenat} % щоб слова переносились закоментуйте цей та наступний рядок
\justifying\sloppy % щоб текст розтягувався
\setlength{\parindent}{0.5in} % виставляємо відступ абзацу
\captionsetup[table]{format=hang,margin=0.5in, justification=raggedright, singlelinecheck=false} % підпис таблиць не по центру а з абзацу

% для форматування картинок та таблиць:
\setlength{\intextsep}{0pt}
\captionsetup[figure]{belowskip=-12pt}
\newlength{\magicspace}\setlength{\magicspace}{2\baselineskip}
% щоб був рядок до і після картинки: 
\let\oldfigure\figure
\let\endoldfigure\endfigure
\renewenvironment{figure}[1][]{\begin{oldfigure}[#1]\vspace{\magicspace}}{\vspace{\magicspace}\end{oldfigure}}
% щоб був рядок до і після таблиці: зроблено експериментально :)
\let\oldtable\table
\let\endoldtable\endtable
\renewenvironment{table}[1][]{\begin{oldtable}[#1]\vspace{\magicspace}}{\vspace{\glueexpr \magicspace - 8pt\relax}\end{oldtable}}
% це капець.....

\addto\extrasukrainian{\renewcommand\refname{Перелік джерел посилання}} % змінити назву переліка посилань з Література на Перелік джерел посилання
% змінюємо нумерацію в переліку посилань з [1] на 1. :
\makeatletter
\renewcommand*\@biblabel[1]{#1.}
\makeatother

\usepackage{totcount}
% каунтер для референсів:
\newtotcounter{citnum}
\def\oldbibitem{} \let\oldbibitem=\bibitem
\def\bibitem{\stepcounter{citnum}\oldbibitem}

% каунтер для рисунків та таблиць:
\usepackage[figure,table]{totalcount}

% каунтер для додатків: (хз як це працює :) )
\AtBeginDocument{
    %% register a counter on the basis of the last chapter in totcounter
    \regtotcounter{section}
}

\usepackage{lastpage} % для каунтера сторінок

\usepackage{enumitem} % для прибирання зайвих відступів поміж ітемів списків
\setlist[]{noitemsep, nolistsep, left=\parindent}
\setlist[itemize]{label=--} % також заміняємо точку на тире
\setlist[enumerate]{label=\arabic*)} % ставимо дужку
% ------------------------------------------- Преамбула закінчилась -------------------------------------------

% ------------------------------------------- Титулка ---------------------------------------------------------
\begin{document}
% Титулка
\thispagestyle{empty}
\begin{center}
Міністерство освіти і науки України\\
Харківський національний університет радіоелектроніки \par
\null\par
Кафедра програмної інженерії \par
\null\par\null\par\null\par
КУРСОВА РОБОТА\\
ПОЯСНЮВАЛЬНА ЗАПИСКА\\
з дисципліни ``Об'єктно-орієнтоване програмування''\\
<НАЗВА> 
\end{center}
\par\null\par\null
\begin{changemargin}{-0.25cm}{0cm}
\begin{tabular}{ p{20em} p{11em} } 
Керівник ,  <посада> & <прізвище, ініціали > \\
Студент гр. <шифр групи> & <прізвище, ініціали > \\
\end{tabular}
\par\null\par\null
\begin{tabular}{ l } 
Комісія: \\
\end{tabular}
\end{changemargin}
\begin{changemargin}{1cm}{0cm}
\begin{tabular}{ p{12em} p{7em} p{8em} }
    <посада> & \underline{\makebox[7em][c]{}} & <прізвище, ініціали> \\
    <посада> & \underline{\makebox[7em][c]{}} & <прізвище, ініціали> \\
    <посада> & \underline{\makebox[7em][c]{}} & <прізвище, ініціали> \\
\end{tabular}
\vspace*{\fill}\end{changemargin}
\begin{center}
Харків -- \the\year{}
\end{center}
\newpage
% ------------------------------------------- Титулка закінчилась ---------------------------------------------

% ------------------------------------------- Аркуш завдання --------------------------------------------------
\begin{center}
    ХАРКІВСЬКИЙ НАЦІОНАЛЬНИЙ УНІВЕРСИТЕТ РАДІОЕЛЕКТРОНІКИ
    \begin{changemargin}{1cm}{0cm}
        \begin{tabular}{ l l }
    Кафедра & \textit{програмної інженерії} \\
Рівень вищої освіти & \textit{перший (бакалаврський)} \\
Дисципліна & \textit{Об’єктно-орієнтоване програмування} \\
Спеціальність & 1\textit{21 Інженерія програмного забезпечення} \\
Освітня програма & \textit{Програмна інженерія} \\
\end{tabular}
    \end{changemargin}
\begin{tabularx}{\textwidth} { 
   >{\raggedright\arraybackslash}X 
   >{\centering\arraybackslash}X 
   >{\raggedleft\arraybackslash}X  }
 Курс \underline{\makebox[5em][c]{\textit{1}}}  & Група \underline{\makebox[5em][c]{\textit{ПЗПІ-23-X}}} & Семестр \underline{\makebox[5em][c]{\textit{2}}}\\
\end{tabularx}
\null\par\null
\textit{\textbf{ЗАВДАННЯ \\
на текстовий проєкт студента}} \par
\underline{\makebox[\textwidth][c]{<прізвише, ім'я, по батькові>}} \\
\scriptsize{(Прізвище, Ім'я, По батькові)} \\
\end{center}
1 Тема проєкту: \\
\uline{\makebox[\textwidth][c]{<тема проєкту>}} \\ 
2 Термін здачі студентом закінченого проекту: \textbf{\textit{``\underline{08}'' - червня - 2024 р.}} \\
3 Вихідні дані до проекту: \par
\uline{<Завдання на курсову роботу> <Завдання на курсову роботу> <Завдання на курсову роботу> <Завдання на курсову роботу> < Завдання на курсову роботу > < Завдання на курсову роботу > <Завдання на курсову роботу>}  \par \null \par \noindent
Зміст розрахунково-пояснювальної записки: \par
\uline{<Вступ, опис вимог, проектування програми, інструкція користувача,
висновки> <Вступ, опис вимог, проектування програми, інструкція,
висновки >} \\
\newpage
% ------------------------------------------- Аркуш завдання закінчився ---------------------------------------

% ------------------------------------------- Календарний план ------------------------------------------------
\noindent
\begin{center}
    КАЛЕНДАРНИЙ ПЛАН \par \null \par
    \end{center}
  \begin{tabular}{|p{1em} | p{17em} | p{11em}|}
     \hline
        \multicolumn{1}{|c|}{\textit{№}} & \multicolumn{1}{c}{\textit{Назва етапу}} & \multicolumn{1}{|c|}{\textit{Термін виконання}} \\ \hline
    1 & Видача теми, узгодження і затвердження теми & 13.02.2024 - 15.03.2024 р. \\ \hline
    2 & Формулювання вимог до програми & \hspace{1em}.\hspace{1em}.2024 – \hspace{1em}.\hspace{1em}.2024 р. \\ \hline
    3 & Розробка підсистеми зберігання та пошуку даних. & \hspace{1em}.\hspace{1em}.2024 – \hspace{1em}.\hspace{1em}.2024 р. \\ \hline
    4 & Розробка функцій \ldots & \hspace{1em}.\hspace{1em}.2024 – \hspace{1em}.\hspace{1em}.2024 р. \\ \hline
    5 & Розробка функцій зберігання та завантаження даних & \hspace{1em}.\hspace{1em}.2024 – \hspace{1em}.\hspace{1em}.2024 р. \\ \hline
    6 & Тестування і доопрацювання розробленої програмної системи. & \hspace{1em}.\hspace{1em}.2024 – \hspace{1em}.\hspace{1em}.2024 р. \\ \hline
    7 & Оформлення пояснювальної записки, додатків, графічного матеріалу & \hspace{1em}.\hspace{1em}.2024 – \hspace{1em}.\hspace{1em}.2024 р. \\ \hline
    8 & Захист & \hspace{0.01em}03.\hspace{0.01em}06.2024 – \hspace{0.01em}08.\hspace{0.01em}06.2024 р. \\ \hline
  \end{tabular}

\par \null \par \null \par \null \par \null \par \null \par \noindent
\begin{tabularx}{\textwidth} { 
   >{\raggedright\arraybackslash}X 
   >{\raggedleft\arraybackslash}X  }
    Cтудент \underline{\hspace{10em}} \\
    \\
    Керівник \underline{\hspace{10em}} & \underline{\hspace{10em}} \\
     & \scriptsize{(Прізвище, Ім'я, По батькові)} \\
     << 21 >> \underline{\makebox[5em][l]{ лютого}} 2024 р.
\end{tabularx}
% ------------------------------------------- Календарний план закінчився -------------------------------------

% ------------------------------------------- Реферат ---------------------------------------------------------
\section*{РЕФЕРАТ}
Пояснювальна записка до курсової роботи: \pageref{LastPage} с., \totalfigures{} рис., \totaltables{} табл., \total{section}
додатк., \total{citnum}\ джер.. \par
ПОКУПЕЦЬ, МАГАЗИН, ЗВІТ, ООП, .NET, МОВА C\# \par
Метою роботи є розробка програми «Довідник покупця», яка буде
надавати користувачу довідки про товари та магазини. \par
В результаті отримана програма, що дозволяє зберігати список магазинів,
характеристики кожного магазину, такі як: назва, спеціалізація, адреса,
телефон, час роботи, вид власності. Є можливість утворювати нові списки
магазинів, додавати, видаляти та редагувати магазини. Також, є функція
формування звіту у Microsoft Excel.  \par
В процесі розробки використано середовище розробки Microsoft Visual
Studio 2022, фреймворк Windows Forms, платформа .NET 8.0, мова
програмування C\#.
% ------------------------------------------- Реферат закінчився ----------------------------------------------

% ------------------------------------------- Зміст -----------------------------------------------------------
\tableofcontents %генерація змісту
% ------------------------------------------- Зміст завершився ------------------------------------------------

% ------------------------------------------- Вступ -----------------------------------------------------------
\section*{Вступ}
\addcontentsline{toc}{section}{Вступ} %додаємо сторінку Вступу до змісту
Ваш ВСТУП
% ------------------------------------------- Вступ закінчився ------------------------------------------------

% ------------------------------------------- Ну і далі сама ваша курсова -------------------------------------
\section{Опис вимог}
\label{sec:requirements} % можна позначати також розділи
Це ваш перший розділ. До нього можна посилатись, якщо в нього є лейбл, наприклад \verb|\label{sec:requirements}|. Тоді при \verb|\ref{sec:requirements}| буде показуватись його номер, тобто \ref{sec:requirements}. Текст не переноситься через строку. Якщо ви хочете увімкнути перенос, то зайдіть у преамбулу.

Це новий параграф. Його можна починати через пусту строку чи через \verb|\par| \par
ніби ви нажали ентер

Для того щоб просто перенести на новий рядок без відступу використовуйте \verb|\\| \\
та текст перенесеться на новий рядок

Щоб робити підрозділ напишіть \verb|\subsection{Підрозділ}| 

\subsection{Підрозділ}
\label{sec:subsection} % можна позначати також розділи
Він автоматично форматується щоб зверху та знизу нього був рядок. Також він автоматично нумерується. До нього можна звертатися як підрозділ \ref{sec:subsection}. Чесно я б їх зробив по-іншому, але в методичці так. Як на мене вони дуже зливаються з іншим текстом.

\subsubsection{Це підпідрозділ, чи пункт. Таке теж буває}

Він як і всі будуть висвічуватись у змісті

\subsection{Рисунки}

Рисунки вставляються ось так:

\begin{figure}[h]
    \centering
    \includegraphics[width=0.5\textwidth]{example-image-a}
    \caption{Підписується правильно, та нумерується також}
    \label{fig:a}
\end{figure}

Все як написано в методичці: зверху та знизу рядок. 

\begin{figure}[h]
    \centering
    \includegraphics[width=0.4\textwidth]{example-image-b}
    \caption{Підписується правильно, та нумерується також}
    \label{fig:b}
\end{figure}

Я можу звертатися до рис. \ref{fig:a} та \ref{fig:b} та не перейматися про їх номери. Картинки плаваючи, тому можуть легко переноситися на інші сторінки. Якщо вам це не подобається, то пошукайте в інеті :). Якщо картинка залізла на наступну сторінку де повинен був починатись новий розділ, то можете зробити \verb|\clearpage| замість \verb|\newpage|. Чи можна зменшити їх ширину.

\subsection{Таблиці}

Таблиці вставляються ось так:

\begin{table}[h]
    \caption{Підписує як треба! з абзацу перед таблицею!}
    \centering
    \begin{tabular}{|c|c|} \hline
        Таблиці & Бувають різними \\ \hline
        Для більшого ознайомлення & \\
        Краще зайти на: & https://www.overleaf.com/learn/latex/Tables \\ \hline
    \end{tabular}
    \label{tab:my_label}
\end{table}

Таблиці також іноді переносяться на іншу сторінку.

\subsection{Cтатистика у рефераті}

Загальна кількість сторінок, рисунків, таблиць, додатків, посилань збирається у першому рядку реферату автоматично (можете туди подивитись). Тому не треба його завжди перероблювати, якщо ви хочете щось додати

\subsection{Посилання}

Посилання оформлюєш так \verb|\cite{wiki-computerscience}|\cite{wiki-computerscience}. Потім додаєте в переліку посилань bibitem з таким же іменем \verb|\bibitem{wiki-computerscience}|. Потім його називаєте за дсту. Наприклад це можна зробити тут\cite{dstu}.

\subsection{Шаблонні розділи}

В шаблонних розділах, які були до цього просто заповнюйте персональну інформацію, яка стосується вашого проєкту та вас безпосередньо. Зазвичай все що вам потрібно змінити знаходится в ось таких <дужках>.

\section{Проєктування програми}

\section{Інструкція користувача}

% ------------------------------------------- Курсова завершилась ---------------------------------------------

% ------------------------------------------- Висновки --------------------------------------------------------
\section*{Висновки}
\addcontentsline{toc}{section}{Висновки} %додаємо сторінку Вступу до змісту
Ваші висновки
% ------------------------------------------- Висновки закінчилися --------------------------------------------

% ------------------------------------------- Перелік джерел посилання ----------------------------------------
\raggedright % увімкнути вирівнювання вліво
\begin{thebibliography}{99} % починаємо перелік та вставляємо туди наші посилання
    \addcontentsline{toc}{section}{Перелік джерел посилання} % додаємо перелік до змісту 
    \bibitem{wiki-computerscience} Contributors to Wikimedia projects. Computer science - Wikipedia. Wikipedia, the free encyclopedia. URL: https://en.wikipedia.org/wiki/Computer\_science (date of access: 09.05.2024).
    \bibitem{dstu}  Grafiati: Оформити списки використаних джерел онлайн. Grafiati: Оформити списки використаних джерел онлайн. URL: https://www.grafiati.com/uk/ (дата звернення: 09.05.2024).
\end{thebibliography}

% ------------------------------------------- Перелік джерел посилання закінчився -----------------------------

% ------------------------------------------- Додатки (якщо треба) --------------------------------------------
\titleformat{\section}[display]{\filcenter}{\MakeUppercase{\bfseries{Додаток \thesection}}}{0pt}{} % зміна форматування заголовків у додатках
\titlecontents{section}[0pt]{\normalfont}{Додаток {\thecontentslabel} }{}{\Dotfill \contentspage} % зміна форматування додатків у змісті
\appendix
\setlength{\parindent}{0.5in}
% додавайте додатки: додатки додаються так само секціями

\section{Вау, це додаток!}
Так, це додаток! Але скоріш за все його ніхто не буде використовувати :'( Дуже сумно.................

\subsection{Це підрозділ! Подивіться на нумеризацію!}

Це класно, ще класніше, якщо і малюнки теж так підписувались \ldots

\begin{figure}[h]
    \centering
    \includegraphics[width=0.5\textwidth]{example-image-a}
    \caption{І підписується првильно!}
    \label{fig:enter-label}
\end{figure}

НІЧОГО СОБІ!!!!!

\section{Вау, це додаток!}
Так, це додаток!
\section{Вау, це додаток!}
Так, це додаток!
\section{Вау, це додаток!}
Так, це додаток!
\section{Бачите як красиво, навіть знає що ґ треба пропускати}
Так, це круто!

% ------------------------------------------- Додатки (якщо треба) закінчилися -------------------------------
\end{document}

